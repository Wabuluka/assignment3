\documentclass{article}
\begin{document}
\title{A research report on HIV/AIDS awareness in Makerere University}
\author{Wabuluka Davies \\ 15/U/13575/PS 215009779}
\maketitle

\section{ Introduction}
\subsection{Background}
\paragraph{In Uganda, the young generation is among the fast growing with more cases of HIV/AIDS reported. In this context, risks and vulnerability are high as the epidemic is on the rise with evidence indicating significantly increasing HIV prevalence, new HIV infections and AIDS-related deaths.}

\subsection{Problem Statement}
\paragraph{University students make up a large contribution to the total number of youth that make up the population of Uganda. Among these is a high prevalence of HIV/AIDS. This brings me to the reason for carrying out this research to know the attitudes of University students about HIV/AIDS and how they try to address the issues that aurround it.}

\subsection{Objective}
\paragraph{The aim of the survey was to assess HIV/AIDS knowledge and attitudes related to HIV/AIDS among a wide group of university students in}

\subsection{Specific Objective}
\paragraph{In particular, the aim of the research was to focus on individual thoughts of the students about the spread of the HIV/AIDs virus in universities.}

\section{Research Scope}
\paragraph{The research was carried out strictly within the boundaries of Makerere University among students from all the Halls of Residence and those got from various schools and colleges in the University.}

\section{ Methodology}
\paragraph{The study was a descriptive cross-sectional survey designed to measure students’ knowledge and attitudes about HIV/AIDS among university students.}
\paragraph{The questionaire was designed with adequate consideeration to students' culture. The research was not conducted to find out about the status of students but rather to collect their attitudes and knowledge about HIV/AIDS.}
\paragraph{The research was carried out using electronic devices such as smartphones and laptop computers together with the relevant softwares. The smartphones where used to collect the data using an application called ODK collect which contained the questionaire. The laptop was used to gather the data from the smartphone and process it so that it is hosted by the Google AppEngine platform cloud.}

\subsection{Procedures}
\paragraph{During the research process, electronic tools such as smartphones and Laptop computers were used. These were enhanced by the relevant softwares that made the data collection process very convinient. The softwares used were ODK collect which was installed on the smartphone and used to directly collect data from the students. }
\paragraph{The data collected using the ODK collect application installed on the smartphones was transfered into the laptop computers which processed it using the ODK aggregate application. This application was used to categorise the data and prepare it for uploading on to the Google AppEngine platform cloud where the whole world can access it.}

\section {Conclusion}
\paragraph{In conclusion during my research, I discovered that although many students are aware of the HIV/AIDS virus they do not really fear the fact that it does not have a cure as yet. Thus those that remember take the right decision to prevent the spread while a few do not care about it.}





\end{document}